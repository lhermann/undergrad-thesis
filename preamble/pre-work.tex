% ------------------------------------------------------------------------
% LaTeX - Preambel ******************************************************
% ------------------------------------------------------------------------
% pre-work
% ========================================================================
% % ToDo kennzeichnen
\newcommand{\workTodo}[1]{\textcolor{red}{todo: #1}}

% % Für Datum und Zeit in Fusszeile
% % !!!Inhalt bei Fertigstellung der Arbeit löschen
% \newcommand{\workMarkDateTime}{\today{} - \thistime}

% % Alle Namen werden im Titel und im hyperref-Paket eingetragen
% % !!! Ueberall für <Wert> das Entsprechende eintragen

 % <Typ> Studienarbeit, Dipolmarbeit, Studienarbeit oder Bachlor-Abschlussarbeit
\newcommand{\workType}{Bachlor-Abschlussarbeit\xspace}

 % <Titel> der Arbeit
\newcommand{\workTitel}{Modularising existing functionality of an enterprise monolith system into microservices\xspace}

 % <Studiengang> z.B. Kommunikationstechnik
\newcommand{\workStudiengang}{Informationstechnik}

% <Semester> mit Jahr z.B. Sommersemester 2008
\newcommand{\workSemester}{Wintersemester 2019/20}

% <Name> des Studenten
\newcommand{\workNameStudent}{Lukas Hermann}

% <Pruefer> Name des prüfenden (betreuenden) Professor an der Hochschule
\newcommand{\workPruefer}{Karin Melzer}


% %%% Nur bei Abschluss-Arbeiten

% <Datum> der Abgabe der Arbeit (Eidesstatliche Erklärung)
\newcommand{\workDate}{\today}

% <Zweitprüfer>
\newcommand{\workZweitPruefer}{Reiner Marchthaler}

% <Zeitraum>
\newcommand{\workTimeframe}{9. Oktober 2019 – 9. Februar 2020}


% %%% Nur bei Industrie-Arbeiten:

% <Firma>
\newcommand{\workComany}{Capgemini Service SAS\xspace}

% <Betreuer in der Firma>
\newcommand{\workBetreuer}{Robin Kurz}

% Firmenlogo Name hier anpassen, Größe (wenn möglich) nicht ändern
\newcommand{\workComanyLogo}{\includegraphics[width=5cm]{assets/logos/capgemini}}
