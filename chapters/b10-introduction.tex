\chapter{Introduction}
\label{sec:intro}

\section{Motivation}
\label{sec:intro-motivation}

Having built a few software applications both before and during the studies in software engineering, the author had a good experience in building an application from the ground up. Thus the interest for this paper lies in the more significant architectural questions of software development. The goal is to learn more about microservices, which is an architectural style very present in the current conversation, and to have a clear grasp on the concept.

I entered the project with several questions that wanted an answer. What is a microservice? When does it make sense to extract a microservice from a monolith and how to make cuts in existing applications? How does a microservice integrate with a monolith or other services? And what are some advantages and challenges the developer is facing with microservice architectures? Another question where the answer turned out to surprising was if microservices are more comfortable to develop than monoliths. In other words, is the shiny new concept better than the old way? It turns out it's complicated.

Besides theoretical questions, it was also the goal to collect the practical experience of actually building a microservice of which the features were part of a monolith before. During the author's career as a software developer, he learned that many concepts could not be understood until one tries it out and gains some firsthand experience with the concept and its pitfalls. The author can happily report that he reached this goal and has gained experience, which helps to make better decisions in the future.

The last goal does not concern itself with the topic of this thesis. It was merely to understand how a big software company operates internally and how an employer experiences such a company. Capgemini provided the opportunity to work on a project within the framework of one of their teams to build a microservice application of one feature of a monolith application for which the author is very thankful.


% \section{Underlying Questions}
% \label{sec:intro-underlying-questions}


\section{About Capgemini}
\label{sec:intro-about-capgemini}

Capgemini is a French publicly traded software company with plants in multiple countries in all parts of the world. In Germany alone, Capgemini has offices in 13 cities. In Stuttgart, the company employs about 300 people.

Capgemini mainly does software projects for clients like the auto industry and has very few in-house projects. The company tries to be remote-friendly with teams of different countries working on the same projects. It cares about the wellbeing of its employees and has a surprisingly young workforce.


\section{Overview}
\label{sec:intro:overview}
