\chapter*{Abstract}

Microservice architectures are gaining popularity because their benefits are even more significant with the rise of cloud computing. How can companies convert their large monolith enterprise systems into microservices, and when does this conversion make sense? This paper investigates how to extract a microservice from an existing monolith through a practical example of an actively used point of sales software. Additionally, the author compared his experience with ideas and methods from technical literature.

The analysis shows that microservices do not result in cheaper or faster development but are a trade-off, increasing complexity in favor of quicker release cycles and better scalability. The process of refactoring a monolith into services requires careful developer communication, disciplined documentation of interfaces, and a dedicated dev-ops team. Estimating the viability and utility of breaking out the functionality into a microservice needs evaluation on a case to case basis.

\begin{figure}[ht]
  \centering
  \includegraphics[width=0.25\linewidth]{assets/illustration-microservice-2.png}
\end{figure}
